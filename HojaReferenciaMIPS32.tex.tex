\documentclass[a4paper,10pt]{article}
\usepackage[utf8]{inputenc}
\usepackage{geometry}
\geometry{margin=1in}
\usepackage{multicol}
\setlength{\columnsep}{1cm}

\title{Hoja de Referencia MIPS32}
\author{Fabian Peña}
\date{}

\begin{document}
\maketitle

\begin{multicols}{2}

\section*{Categorías de Instrucciones}

\subsection*{1. Carga y Almacenamiento}

\begin{itemize}
    \item \textbf{lw} \$t0, offset(\$t1) \hfill Carga palabra desde memoria
    \item \textbf{sw} \$t0, offset(\$t1) \hfill Almacena palabra en memoria
    \item \textbf{lb} / \textbf{sb} \hfill Carga/almacena byte (con signo)
    \item \textbf{lbu} \hfill Carga byte sin signo
    \item \textbf{la} \$t0, etiqueta \hfill Carga dirección
\end{itemize}

\subsection*{2. Aritméticas}

\begin{itemize}
    \item \textbf{add} \$t0, \$t1, \$t2 \hfill Suma con overflow
    \item \textbf{addu} \$t0, \$t1, \$t2 \hfill Suma sin overflow
    \item \textbf{sub} \$t0, \$t1, \$t2 \hfill Resta con overflow
    \item \textbf{subu} \$t0, \$t1, \$t2 \hfill Resta sin overflow
    \item \textbf{mult} \$t1, \$t2 \hfill Multiplica (resultado en HI/LO)
    \item \textbf{mflo} \$t0 \hfill Copia parte baja del resultado
    \item \textbf{mfhi} \$t0 \hfill Copia parte alta del resultado
    \item \textbf{mul} \$t0, \$t1, \$t2 \hfill Pseudo-instrucción, usa HI/LO
    \item \textbf{div} \$t0, \$t1 \hfill División
\end{itemize}

\subsection*{3. Lógicas}

\begin{itemize}
    \item \textbf{and} \$t0, \$t1, \$t2 \hfill AND bit a bit
    \item \textbf{or} \$t0, \$t1, \$t2 \hfill OR bit a bit
    \item \textbf{xor} \$t0, \$t1, \$t2 \hfill XOR bit a bit
    \item \textbf{nor} \$t0, \$t1, \$t2 \hfill NOR bit a bit
    \item \textbf{li} \$t0, valor \hfill Cargar inmediato
    \item \textbf{move} \$t0, \$t1 \hfill Copiar registros
\end{itemize}

\subsection*{4. Desplazamientos}

\begin{itemize}
    \item \textbf{sll} \$t0, \$t1, n \hfill Desplazamiento lógico izq.
    \item \textbf{srl} \$t0, \$t1, n \hfill Desplazamiento lógico der.
    \item \textbf{sra} \$t0, \$t1, n \hfill Desplazamiento aritmético der.
\end{itemize}

\subsection*{5. Control de Flujo}

\begin{itemize}
    \item \textbf{beq} \$t0, \$t1, etiqueta \hfill Salta si son iguales
    \item \textbf{bne} \$t0, \$t1, etiqueta \hfill Salta si son distintos
    \item \textbf{j} etiqueta \hfill Salto incondicional
    \item \textbf{jal} etiqueta \hfill Salta y guarda dirección de retorno
    \item \textbf{jr} \$ra \hfill Retorna a la dirección guardada
\end{itemize}

\subsection*{6. Syscalls}

\begin{itemize}
    \item \textbf{li} \$v0, 1 \hfill Imprimir entero \\
          \ \ \$a0 = entero
    \item \textbf{li} \$v0, 4 \hfill Imprimir cadena \\
          \ \ \$a0 = dirección del texto
    \item \textbf{li} \$v0, 5 \hfill Leer entero (resultado en \$v0)
    \item \textbf{li} \$v0, 10 \hfill Terminar programa
\end{itemize}

\end{multicols}
\end{document}
